\section{Lattice models for phase transitions}

A very successfull and general rage of  models in statistical mechanics are the
ones where the components of a system are allocated on an array of lattice sites,
where every component interacts only wiht the closest ones to it 
(\emph{nearest-neighbour interaction}).
This kind of models turns out to describe well a very broad range of phase 
transitions including ferromagnetism transitions, gas-liquid and gas-solid 
transitions, superconductive transitions and so on. \footnote{The foregoing
discussion is heavily based on ~\textcite{pathria1972statistical} section 12.3}

The case in consideration here will be the one for \emph{ferromagnetic}
transitions.

Being defined on a lattice means basically a discretization of space, a caveat 
that fit very well if the system in consideration is a metal whose atoms are 
situated in a crystal structure. The crystalline nature of metals enables to
introduce another semplification in this kind of models: considering the atoms
as harmonic oscillators swingind around the equilibrium point we can neglect the
kinetic energy of these atoms and considering only the interactions between them.
It has to be said that this kind of facilitation turns out to be good enough also
for the other kinds of transitions listed above.

\section{Hamiltonian of ferromagnetic lattice systems}

Being interested in the magnetic interactions of atoms we consider them as a 
magnetic dipoles having a magnetic moment $\mathbf{\mu}$. Every magnetic dipole
in the $N$ lattice sites will have a magnitude $\mu = g \mu_B \sqrt{J(J+1)}$, 
with $g$ being the Landè factor, $\mu_B$ the Bohr magneton and $J$ being the
total angolar momentum quantum number. The orientations of the magnetic momenta
is given by the multiplicity of $J$: $(2J+1)$. This magnetic dipoles will interact
witch each other and will interact with an eventual external field $\mathbf{B}$.

The net magnetization of the system $\overline{M}$ will depend on both the 
temperature $T$ and the field $B$ ($\overline{M} \equiv \overline{M}(T,B)$). 
Studying the \emph{spontaneous magnetization} $\overline{M}(0,B)$ of
a \emph{critical temperature} $T_c$ can be defined as the threshold temperature
for $\overline{M}(0,B) \neq 0$: above $T_c$ the termal agitation will be to
big to allow a spontaneous magnetization to form, then at $T_c$ the system will
perform a \emph{ferromagnetic transition}.

Detailed studies beyond the scope of this tractation have shown that ferromagnetic
properties arise when $J=\frac{1}{2}$ and so it can be assumed that this kind of 
phenomena are due to electons spin. Then we can rewrite the magnetic moment as 
$\mu = 2 \mu_B \sqrt{s(s+1)}$, where $g=2$ is the Landè factor for electrons and
$s$ is the spin of the electron.

From quantum mechanics considerations \parencite[see][]{bransden2003physics}
it can be shown that the interaction energy between two electrons can be expressed
as $K_{ij} \pm J_{ij}$ with $i$ and $j$ being the two indices indicating two
neighboring electrons. The plus and minus sign are determined based on $S$ the 
total spin of the the two electron: the plus corresponding to the case $S=0$ and 
the minus to one with $S=1$, where only $J_{ij}$ is due to the spin configuration.

Then the energy difference between the "parallel" spin configuration and the 
"antiparallel" spin configuration is

$$\Delta E = E_{\uparrow\uparrow} - E_{\uparrow\downarrow} = -2J_{ij}$$

Defining the scalar product operator of the spin of the two electron as 

$$ \mathbf{s_j} \cdot \mathbf{s_i} = \frac{1}{2} S(S+1) - s(s+1)$$

is easy to show that the interaction energy between the spins can be written as

$$ E_{ij} = \text{const.} - 2J_{ij} (\mathbf{s_i} \cdot \mathbf{s_j})$$

The exact value of the constant is irrilevant since any constant can be added to
the potential energy. Still from quantum meachanical considerations can be seen
$J_{ij}$ fall very rapidly with the distance, supporting our initial caveat of
regarding only the neirest-neighbors interactions.


\section{\textit{n}-vector model}

\subsection{Ising Model}

\subsection{Heisenberg Model}


