\section{Lattice models for phase transitions}

A very successful and general set of models in statistical mechanics is the
one in which the components of a system are located on an array of lattice sites,
with every component interacting only with the closest ones
(\emph{nearest-neighbour interaction}).
This kind of models turns out to describe well enough a very broad range of phase 
transitions including ferromagnetism transitions, gas-liquid and gas-solid 
transitions, superconductive transitions and so on. \footnote{The foregoing
discussion is heavily based on ~\textcite{pathria1972statistical} section 12.3}

The case considered in this dissertation will be the \emph{ferromagnetic}
transitions.

Being defined on a lattice entails a discretization of space, a caveat 
that fits very well if the system under study is a metal whose atoms are almost 
locked in a crystalline structure. The crystalline nature of metals allows to
introduce another semplification in this kind of models: considering the atoms
as harmonic oscillators swinging around the equilibrium point, their kinetic energy can 
be neglected and only their magnetic interactions has to be considered. 
It has to be said that this kind of facilitation turns out to be 
adequate also for other kinds of transitions and for materials very different 
from metals.

\section{Hamiltonian of ferromagnetic lattice systems}

Being interested only in the magnetic interactions of atoms means that they can be considered as 
magnetic dipoles having a magnetic moment $\mathbf{\mu}$. Every magnetic dipole
in the $N$ lattice sites will have a magnitude $\mu = g \mu_B \sqrt{J(J+1)}$, 
with $g$ being the Landè factor, $\mu_B$ the Bohr magneton and $J$ being the
total angolar momentum quantum number. The total number of possible orientations of
the magnetic momenta in space is given by the multiplicity of $J$: $(2J+1)$. This 
magnetic dipoles will interact witch each other and will interact with an eventual
external field $\mathbf{B}$.

The net magnetization of the system $\overline{M}$ will depend on both the 
temperature $T$ and the field $B$ ($\overline{M} \equiv \overline{M}(T,B)$). 
Studying the \emph{spontaneous magnetization} $\overline{M}(T,0)$ of the system,
a \emph{critical temperature} $T_c$ can be defined as the threshold temperature
for $\overline{M}(T_c,0) \neq 0$ for $T < T_c$: above $T_c$ the thermal agitation 
will be too big to allow a spontaneous magnetization to manifest, then at $T_c$ the
system will perform a \emph{ferromagnetic transition}.

Detailed studies beyond the scope of this dissertation have shown that ferromagnetic
properties arise when $J=\frac{1}{2}$ and so it can be assumed that this kind of 
phenomena are due to electrons' spin. Then the magnetic moment can be rewritten as 
$\mu = 2 \mu_B \sqrt{s(s+1)}$, where $g=2$ is the Landè factor for electrons and
$s$ is the spin of the electron.

From quantum mechanics considerations \parencite[see][]{bransden2003physics}
it can be shown that the interaction energy between two electrons can be expressed
as $K_{ij} \pm J_{ij}$ with $i$ and $j$ being the two indices indicating two
neighbouring electrons. The plus and minus sign are determined based on $S$, the 
total spin of the the two electron: the plus corresponding to the case $S=0$ and 
the minus to one with $S=1$, where only $J_{ij}$ depends on the spins' configuration.

Then the energy difference between the "parallel" spin configuration and the 
"antiparallel" spin configuration is
$$\Delta E = E_{\uparrow\uparrow} - E_{\uparrow\downarrow} = -2J_{ij}$$
Defining the scalar product operator of the spin of the two electron as 
$$ \mathbf{s}_j \cdot \mathbf{s}_i = \frac{1}{2} S(S+1) - s(s+1)$$
is easy to show that the interaction energy between the spins can be written as
$$ E_{ij} = \text{const.} - 2J_{ij} (\mathbf{s}_i \cdot \mathbf{s}_j)$$

The exact value of the constant is irrilevant since any constant can be added to
the potential energy. Still from quantum mechanical considerations it can be seen
that $J_{ij}$ falls very rapidly when the distance beetwen the two spins increases,
supporting our initial caveat of regarding only the neirest-neighbours interactions.


\section{\textit{n}-vector models}

In statistical mechanics the formalization of the ideas presented in the previous
sections for ferromagnetic lattice models is the so called \textit{n}-vector model.
A \textit{n}-vector model is a system of interacting spins on a crystalline lattice 
as the ones described above. This formalization was developed by Stanley
\parencite[see][]{PhysRevLett.20.589} who was able to show that most of the
critical properties of this kind of systems depends only on the dimension of the
lattice and on the dimensionality of the spins in the system.

Fromalized in a mathematical way we have a \textit{n}-vector model when we can 
describe a system composed of \textit{n}-dimensional spins as unit-length vectors,
located on the sites of $d$ dimensional lattice.

The evolution of the system is described by the hamiltonian
\begin{equation}
	H = - J \sum_{n.n.} \mathbf{s}_i \cdot \mathbf{s}_j
	\label{eq:ham_nvector}
\end{equation}
where the meaning of the symbols follows directly from the one used in the
previous sections and the subscript $n.n.$ means the summation goes over only
on the nearest-neighbour spins. 

Equation~\ref{eq:ham_nvector} shows that \emph{ferromagnetic} phenomena are  
possible when $J > 0$ since the configuration $\uparrow\uparrow$ is 
energetically favorable in respect to the $\uparrow\downarrow$ configuration; 
if $J < 0$ \emph{antiferromagnetic} phenomena could arise.

Obviously in the presence of an external field the hamiltonian will be modified
in the following way
\begin{equation}
\label{eq:ham_field}
H = - J \sum_{n.n.} \mathbf{s}_i \cdot \mathbf{s}_j - h \sum_i \mathbf{s}_i 
\end{equation}
where now $J$ is the same as in equation~\ref{eq:ham_nvector} divided by $k_B T$ 
and $h = \mu B / k_B T$, where $B$ is the magnitude of the external magnetic field. 
As a consequence of the presence of the field, magnetization properties will also 
depend on the field itself.


\subsection{Ising Model}

The Ising model was the first of the ferromagnetic lattice models invented, much 
before Stanley's formalization, in 1920 by professor Wilhelm Lenz who gave it as 
problem to his student Ernst Ising. \footnote{check \url{
https://en.wikipedia.org/wiki/Ising_model}}

The Ising model is basically the \textit{n}-vector model for $n=1$: the spins
are considered as a one dimesional vector that can take only two values: up and 
down. This is the simplest physically important case ($n=0$ being of importance merely
from a mathematical point of view) and was solved analytically in the $d=1$ case by
Ising in 1924. He showed no transitions can occur in such a model, but it was the
ground zero for the development of much more complicated models for higher
dimensions.

In the following years the case for $d=2$ was also solved and transitions were
shown to be possible. Analytical solutions for higher dimension are still to be
found, if there are some, and are definitely outside of the scope of this
dissertation. 


\subsection{Heisenberg Model}

The Heisenberg model is the \textit{n}-vector model for $n=3$ and require a quantum
mechanical approach and is obviously the closest one also to a description of reality.
Being more complex as a model makes it a tremendously more difficult problem to 
tackle and it requires a mathematical and quantum mechanics formalization called 
Potts models \footnote{check \url{https://en.wikipedia.org/wiki/Potts_model}} that
integrates also the \textit{n}-vector models.

The Heisenberg model has multiple apllications in quantum mechanics and it is one
of the most studied models of magnetism. \footnote{For a detailed view of Heisenberg
model see~\textcite{Nolting2009}} 
