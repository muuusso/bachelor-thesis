A theorical analisys of the XY model has been given, starting from its
generalization the n-vector model, simpler cases have been described such as the
Ising model. A detailed theorical study of the 2D and 3D cases for the XY model have been
presented. Monte Carlo simulation theory has been breafly described and has been
put in practice for the 3D case of the XY model. In the last chapter simulation results
have been presented.

The results in the third chapter are not good enough to be compared to other
studies: this is probably due to the fact that near the critical point Monte
Carlo simulations tends not to perform greatly. They tend up to be stuck in
local minima of energy and oscillations in those region of temperature are so high
the averaging process of the algorithm perform much worse than it does in other
regions. This behaviour is known and there exist a few techniques to avoid this
from happening. An example could be the over-relaxation method where the value of
a spin is changed with a different one resulting in the same value of energy: the
sampler will be oblied to sweep more of the phase space and the likelyhood of 
being stuck in a local minimum will decrease significantly. To reduce the
effects of oscillations, averages on slightly different values of temperature
could be taken e.g. the values of observables at $T=2.2$ could be considered 
as the average of the values at $T = 2.195$, $T=2.196$ ... $T=2.205$.\footnote{for
a very detailed work of a Monte Carlo simulation on a GPU for the 3D XY model 
see \cite{Lan}}

All of these technique are generally computational expensive and the limited
equipment did not allow to use them. Anyway in the previous chapters the 
macroscopic behaviour of the XY model has been shown, and even if the values of
critical exponents are not good enough to be considered as a good estimate, 
they are good enough to be compared with some experimental values.


